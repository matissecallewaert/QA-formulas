\documentclass{article}
\usepackage{amsmath}
\usepackage{amssymb}
\usepackage{array}
\usepackage{geometry}
\usepackage{longtable}
\geometry{a4paper, margin=1in}

\title{Formularium}
\author{}
\date{}

\begin{document}

\maketitle

\section{Probability theory}
\renewcommand{\arraystretch}{2} % Increases row height for better readability
\setlength{\tabcolsep}{8pt}
\begin{longtable}{| p{7cm} | p{8cm} |}
    \hline
    \textbf{Formula} & \textbf{Description} \\ \hline
    \endfirsthead
    \hline
    \textbf{Formula} & \textbf{Description} \\ \hline
    \endhead
    \hline
    \endfoot
    $F_X(x)= Pr[x \le x]$ & Cumulative distribution function of a random variable, this is non-decreasing and right-continues \\
    \hline
    $\lim_{x -> \infty} F_X(x) = 1$ & The normalization condition of this CDF \\
    \hline
    $\textnormal{f}_X(x) = \frac{dF_X(x)}{dx} ;\quad \textnormal{f}_X(x)dx = Pr[x < X \le x +dx]$ & The density function, the derivative of the CDF \\
    \hline
    $\int_{0}^{\infty}\textnormal{f}_X(x)dx = 1$ & The normalization condition of the density function \\
    \hline
    $ p_X(n) = Pr[X = n], n \in \mathbb{N} $ & The probability mass function (discrete RV) \\
    \hline
    $ \sum_{n=0}^{\infty} p_X(n) = 1 $ & The normalization function (discrete RV) \\
    \hline
    $ F_{XY}(x,y) = Pr[X \le x, Y \le y] \quad x, y \in \mathbb{R}_{\ge 0} $ & joint cumulative distribution function of two random variables \\
    \hline
    $ Pr[X \le x, Y \le y] \ne Pr[X \le x] Pr[Y \le y]$, if they are independent RV it is equal, thus $ F_{XY}(x, y) = F_X(x)F_Y(y) ;\quad F_X(x) = F_{XY}(x,\infty);\quad F_Y(y) = F_{XY}(\infty, y) $ & properties of the joint CDF \\
    \hline
    $ f_{XY}(x,y) = \frac{\partial^2 F(x, y)}{\partial x \partial y} $ and have that $ f_{XY}(x,y)dxdy = Pr[x < X \le x +dx, y< Y \le y + dy] $ & Joint density function \\
    \hline
    $ p_{XY} (n,m) = Pr[X = n, Y = m] $ & joint mass function (discrete RV) \\
    \hline
    $ f_{X|Y}(x|y) = \frac{f_{XY}(x, y)}{f_Y(y)} $ and have that $ f_{X|Y}(x|y)dx = Pr[x < X \le x + dx | Y = y] $& Conditional density function \\
    \hline
    $ p_{X|Y}(n|m) = Pr[x = n | Y = m] = \frac{p_{XY}(n,m)}{p_Y(m)} $ & Conditional mass function (discrete RV) \\
    \hline
    $ E[X] = \int_{0}^{\infty} x dF_X(x) = \int_{0}^{\infty} x f_X(x)dx $ & Mean or expected value of a RV, it is the summary of a complete probability distribution \\
    \hline
    $ E[X] = \sum_{n=0}^{\infty} n p_X (n)$ & Mean or expected value of a discrete RV, it is the summary of a complete probability distribution \\
    \hline
    $ E[aX + bY] = a E[X] + b E[Y]$ & $E[.]$ is a linear operator \\
    \hline
    $ E[XY] = E[X]E[Y]$ & If RV X and Y are independent \\
    \hline
    $ Var[X] = E[(X - E[X])^2] = E[X^2] - E[X]^2$ & The variance of a RV \\
    \hline
    $ E[X] = E[E[X|Y]]$ & The conditional expectation of X given Y \\
    \hline
\end{longtable}

\section{Terminology}

\begin{longtable}{| p{7cm} | p{8cm} |}
    \hline
    \textbf{Name} & \textbf{Description}  \\ \hline
    \endhead
    \hline
    \endfoot
	Arrival instant & The time at which a customer arrives at the queue. \\
    \hline
    Service instant & The time at which a customer leaves the system after being served completely. \\
    \hline
    Queue content & The number of customers in the queue waiting for service. \\
    \hline
    System content & The number of customers in the total system. \\
    \hline
    Queue capacity & The maximum number of customers in the queue. \\
    \hline
    System capacity & The maximum number of customers in the system. \\
    \hline
    Service time & The amount of time that the customer occupies a server. \\
    \hline
    Waiting time & The amount of time a customer waits in the queue before starting service. \\
    \hline
    Delay or sojourn time & The amount of time a customer resides in the system. \\
    \hline
\end{longtable}

\section{Distributions}
\renewcommand{\arraystretch}{2} % Increases row height for better readability
\setlength{\tabcolsep}{8pt}
\begin{longtable}{| p{3cm} | p{7cm} | p{2cm} | p{2cm} |}
    \hline
    \textbf{Name} & \textbf{Density Function} & \textbf{Mean} & \textbf{Variance} \\ \hline
    \endhead
    \hline
    \endfoot
    Binomial & $ Pr[X = k] = \binom{n}{k} p^k(1 - p)^{n-k} $ with $\binom{n}{k} = \frac{n!}{k!(n-k)!} $ & $np$ & $np(1-p)$\\
    \hline
    Geometric & $ Pr[X = k] = (1 - p)^{k-1}p $ & $\frac{1}{p}$ & $ \frac{1-p}{p^2} $\\
    \hline
    Normal & $ f(x|\mu, \sigma^2) = \frac{1}{\sqrt{2 \pi \sigma^2}} e^{-\frac{(x - \mu)^2}{2 \sigma^2}} $ & $\mu$ & $ \sigma^2 $\\
    \hline
    Uniform & $ f(x|a, b) = {\begin{cases}{\frac {1}{b-a}}\quad {\text{for }}a\leq x\leq b,\\[8pt]0 \quad{\text{for }}x<a\ {\text{ or }}\ x>b.\end{cases}} $ & $\frac{a + b}{2} $ & $\frac{(b - a)^2}{12}$ \\
    \hline
    Exponential & $ f(x|\lambda) = \lambda e^{- \lambda x}$ & $\frac{1}{\lambda}$ & $\frac{1}{\lambda^2}$ \\
    \hline
    Poisson & $ f(k|\lambda) = \frac{e^{-\lambda}\lambda^k}{k!}$ & $\lambda$ & $\lambda$ \\
    \hline
    Erlang & $ f_N(x) = \frac{\mu^N x^{N-1} e^{-\mu x}}{(N - 1)!}$ & $N\frac{1}{\mu}$ & $N\frac{1}{\mu^2}$ \\
    \hline
\end{longtable}

\subsection{Memoryless property}

\begin{longtable}{| p{7cm} | p{8cm} |}
    \hline
    \textbf{Formula} & \textbf{Description}  \\ \hline
    \endhead
    \hline
    \endfoot
	$ r(t) = \frac{Pr[X \le t +dt | X > t]}{dt} = \frac{x(t)}{1 - X(t)}$ & The hazard rate function of a RV \\
    \hline
    $ r(t) = \lambda $ & The hazard rate function of an exponential function \\
    \hline
\end{longtable}

\section{Markov property}

\begin{longtable}{| p{7cm} | p{8cm} |}
    \hline
    \textbf{Formula} & \textbf{Description}  \\ \hline
    \endhead
    \hline
    \endfoot
	$ Pr[X(t) = n | X(t_1)=n_1, \dots, X(t_k)=n_k] = Pr[X(t) = n|X(t_k)=n_k] $ & The Markov property, stating that the futur state only depends on the present state and not on the past states. \\
    \hline
\end{longtable}

\section{M/M/1 queue}

\begin{longtable}{| p{7cm} | p{8cm} |}
    \hline
    \textbf{Formula} & \textbf{Description}  \\ \hline
    \endhead
    \hline
    \endfoot
	$ \lambda s(n) = \mu s(n + 1), \quad n = 0,1,2,\dots $ & The local balance equations for the M/M/1 queue\\
    \hline
    $ s(n) = \rho^n s(0) $ with $ \rho = \frac{\lambda}{\mu} $ & Recursive solution of the balance equations, with $\rho$ the load on the system. \\
    \hline
    $ \sum_{n=0}^{\infty} s(n) = 1 = \sum_{n=0}^{\infty} \rho^n s(0) = \frac{s(0)}{1 - \rho} $ & Using the normalization function, giving us a solution for $s(0)$ \\
    \hline
    $ s(0) = 1 - \rho $ & The solution for $s(0)$. \\
    \hline
    $ s(n) = (1 - \rho) \rho^n $ & The solution for $s(n)$. \\
    \hline
    $ E[S] = \frac{\rho}{1 - \rho} $ & The mean system content of an M/M/1 queue. \\
    \hline
    $ Var[S] = \frac{\rho}{(1 - \rho)^2} $ & The variance system content of an M/M/1 queue. \\
    \hline
    $ q(0) = s(0) + s(1) $ & The queue content, when no customer in the queue, we have either situation where there is no customer in the system or one customer that is being served. \\
    \hline
    $ q(n) = s(n+1), \quad n=1,2,\dots $ & When there are customers in the system, the queue has one less than the system. \\
    \hline
    $ q(0) = 1 - \rho^2, \quad q(n) = (1 - \rho) \rho^{n+1} \quad n=1,2,\dots$ & Filled in using previous formulas. \\
    \hline
    $ E[Q] = \frac{\rho^2}{1 - \rho} $ & The mean queue content. \\
    \hline
    $ Var[Q] = \frac{\rho^2}{(1 - \rho)^2}(1 + \rho - \rho^2) $ & The variance of the queue content. \\
    \hline
    $ E[in] = E[out] $ & Operational law \\
    \hline
\end{longtable}

\section{The generic BD-Queueing system}
\renewcommand{\arraystretch}{3} % Increases row height for better readability
\begin{longtable}{| p{9cm} | p{6cm} |}
    \hline
    \textbf{Formula} & \textbf{Description}  \\ \hline
    \endhead
    \hline
    \endfoot
	$ \lambda_n s(n) = \mu_{n+1} s(n + 1) $ & The local balance equations for a generic Birth-death queue.\\
    \hline
    $ s(n) = s(0) \prod_{m=0}^{n-1} \frac{\lambda_m}{\mu_{m+1}} $ & The local balance equations recursively solved.\\
    \hline
    $ \sum_{n=0}^{\infty}s(n) = 1 = s(0)(1 + \sum_{n=1}^{\infty}\prod_{m = 0}^{n-1}\frac{\lambda_m}{\mu_{m+1}}) $ & Using the normalization property, we get an extra equation. \\
    \hline
    $ s(0) = (1 + \sum_{l=1}^{\infty}\prod_{m = 0}^{l-1}\frac{\lambda_m}{\mu_{m+1}})^{-1} $ & Solution for s(0). \\
    \hline
    $ s(n) = \frac{\prod_{m=0}^{n-1}\frac{\lambda_m}{\mu_{m+1}}}{1 + \sum_{l=1}^{\infty}\prod_{m = 0}^{l-1}\frac{\lambda_m}{\mu_{m+1}}} $ & Solution for s(n). \\
    \hline
    $ E[S] = \sum_{n=1}^{\infty} n \frac{\prod_{m=0}^{n-1}\frac{\lambda_m}{\mu_{m+1}}}{1 + \sum_{l=1}^{\infty}\prod_{m = 0}^{l-1}\frac{\lambda_m}{\mu_{m+1}}} $ & Mean system content. \\
    \hline
    $ Var[S] = \sum_{n=1}^{\infty} n^2 \frac{\prod_{m=0}^{n-1}\frac{\lambda_m}{\mu_{m+1}}}{1 + \sum_{l=1}^{\infty}\prod_{m = 0}^{l-1}\frac{\lambda_m}{\mu_{m+1}}} - (\sum_{n=1}^{\infty} n \frac{\prod_{m=0}^{n-1}\frac{\lambda_m}{\mu_{m+1}}}{1 + \sum_{l=1}^{\infty}\prod_{m = 0}^{l-1}\frac{\lambda_m}{\mu_{m+1}}} )^2 $ & Mean system content. \\
    \hline
    $ \sum_{n=1}^{\infty} \prod_{m=0}^{n-1} \frac{\lambda_m}{\mu_{m+1}} $ & Stability condition, this converges when the system is stable. \\
    \hline
\end{longtable}

\newpage
\section{The M/M/c queue}
\renewcommand{\arraystretch}{3} % Increases row height for better readability
\begin{longtable}{| p{9cm} | p{6cm} |}
    \hline
    \textbf{Formula} & \textbf{Description}  \\ \hline
    \endhead
    \hline
    \endfoot
    $ s(n) = {\begin{cases}{s(0)\frac {(c \rho)^n}{n!}}\quad {\text{for }} n\le c ,\\[8pt] s(0) \frac{(c\rho)^c}{c!}\rho^{n-c} \quad{\text{for }} n>c.\end{cases}} $ & System content probabilities. \\
    \hline
    $ s(0) = (\sum_{l=0}^{c-1}\frac{(c\rho)^l}{l!} + \frac{(c\rho)^c}{c!}\frac{1}{1- \rho}) $ & Normalization condition.\\
    \hline
    $ \rho = \frac{\lambda}{c*\mu} < 1 $ & Stability condition. \\
    \hline
    $ p^Q  = (\sum_{l=0}^{c-1}\frac{(c\rho)^l}{l!} + \frac{(c\rho)^c}{c!}\frac{1}{1- \rho})^{-1} \frac{(c\rho)^c}{c!}\frac{1}{1 - \rho} $ & Erlang C formula. It is the probability that an arriving customer has to wait. \\
    \hline
    $ E[Q] = p^Q \frac{\rho}{1 - \rho} $ & Mean queue content. \\
    \hline
    $ E[S] = p^Q \frac{\rho}{1 - \rho} + c\rho $ & Mean system content. \\
    \hline
    $ c\rho $ & Average number of customer being served in the system. \\
    \hline
\end{longtable}

\section{The M/M/1/K queue}

\begin{longtable}{| p{9cm} | p{6cm} |}
    \hline
    \textbf{Formula} & \textbf{Description}  \\ \hline
    \endhead
    \hline
    \endfoot
    $ n = 1, \ldots, K $ & For all formulas. \\
    \hline
    $ s(n) = \frac{1 - \rho}{1 - \rho^{K+1}} \rho^n $ & System content probabilities. \\
    \hline
    $ E[S] = \frac{\rho}{1 - \rho} \frac{1 - (K(1-\rho) + 1)\rho^K}{1 - \rho^{K+1}} $ & System content probabilities. \\
    \hline
    $ l_p = \frac{1 - \rho}{1 - \rho^{K+1}} \rho^K $ & Loss rate or loss probability. \\
    \hline
    $ \lambda(1 - l_p) = \mu(1 - s(0)) $ & The operational law. \\
    \hline
    $ l_p = 1 - \frac{1 - s(0)}{\rho} $ with $ \rho \ge 1 \rightarrow l_p \ge 1 - \frac{1}{\rho} $ & Alternative solution for the loss. \\
    \hline
    $ \lambda_{eff} = \lambda(1 - l_p) $ and $ \rho_{eff} = \rho (1 - l_p) $ & The effective arrival rate and effective load. \\
    \hline
\end{longtable}

\section{Waiting time and delay}

\begin{longtable}{| p{9cm} | p{6cm} |}
    \hline
    \textbf{Formula} & \textbf{Description}  \\ \hline
    \endhead
    \hline
    \endfoot
    $ D = \sum_{n=1}^{S^A + 1} B_n $ & The delay is equal to all the service times of the customers in the system when the tagged customer arrives plus the service time of that tagged customer. \\
    \hline
    $ Pr[D \le t] = E[Pr[D \le t | S^A]] = \sum_{m=0}^{\infty} Pr[D \le t | S^A = m] Pr[S^A = m] $ & The law of total probability. \\
    \hline
    $ Pr[D \le t | S^A] = 1 - \sum_{n=0}^{S^A} \frac{1}{n!} e^{- \mu t} (\mu t)^n  $ & Delay is Erlang distributed with $ S^A + 1 $ stages and rate $\mu$ \\
    \hline
    $ Pr[S^A = m] = s(m) = (1 - \rho) \rho^m $ & Using PASTA for the stationary distribution of the system. \\
    \hline
    $ Pr[D \le t] = 1 - e^{-(\mu - \lambda)t} $ & Going further on the law of total probability we derive that the delay is exponentially distributed with rate $\mu - \lambda$. So Delay is memoryless. \\
    \hline
    $ E[D] = \frac{1}{\mu - \lambda} $ & The mean delay. \\
    \hline
    $ Var[D] = \frac{1}{(\mu - \lambda)^2} $ & The variance of the delay. \\
    \hline
    $ E[D] = \frac{1}{\lambda} E[S] $ & Little's law. \\
    \hline
    $ E[W] = \frac{1}{\lambda}E[Q] $ & The mean waiting time in relation to the mean queue content. \\
    \hline
    $ Var[W] = \frac{\rho(2-\rho)}{(\mu - \lambda)^2} $ & The variance of the waiting time. \\
    \hline
\end{longtable}

\section{Little's law}

Little's law states that the average number of customers in a stationary system is equal to the average arrival rate times the average time a customer spends in the system.

This is valid for all queueing systems! For $\lambda < \infty$, $ \overline{D} < \infty $ and $ \overline{S} < \infty $.

\[ \overline{S} = \lambda \overline{D} \]

Make sure that for example in finite systems, the effective arrival rate $\lambda_{eff}$ is used.
\end{document}